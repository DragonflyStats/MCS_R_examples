 \documentclass[a4paper,12pt]{article}
%%%%%%%%%%%%%%%%%%%%%%%%%%%%%%%%%%%%%%%%%%%%%%%%%%%%%%%%%%%%%%%%%%%%%%%%%%%%%%%%%%%%%%%%%%%%%%%%%%%%%%%%%%%%%%%%%%%%%%%%%%%%%%%%%%%%%%%%%%%%%%%%%%%%%%%%%%%%%%%%%%%%%%%%%%%%%%%%%%%%%%%%%%%%%%%%%%%%%%%%%%%%%%%%%%%%%%%%%%%%%%%%%%%%%%%%%%%%%%%%%%%%%%%%%%%%
\usepackage{eurosym}
\usepackage{vmargin}
\usepackage{amsmath}
\usepackage{multicol}
\usepackage{graphics}
\usepackage{epsfig}
\usepackage{framed}
\usepackage{subfigure}
\usepackage{fancyhdr}

\setcounter{MaxMatrixCols}{10}
%TCIDATA{OutputFilter=LATEX.DLL}
%TCIDATA{Version=5.00.0.2570}
%TCIDATA{<META NAME="SaveForMode" CONTENT="1">}
%TCIDATA{LastRevised=Wednesday, February 23, 2011 13:24:34}
%TCIDATA{<META NAME="GraphicsSave" CONTENT="32">}
%TCIDATA{Language=American English}

%\pagestyle{fancy}
\setmarginsrb{20mm}{0mm}{20mm}{25mm}{12mm}{11mm}{0mm}{11mm}
%\lhead{MA4413 2013} \rhead{Mr. Kevin O'Brien}
%\chead{Midterm Assessment 1 }
%\input{tcilatex}

\begin{document}




\begin{enumerate}

\item 
Find the probability of a "z" random variable being between -1.8 and 1.96?

\begin{framed}
Solution 
\begin{itemize}


\item Consider the complment event of being in this interval: a combination of being too low or two high. 

\item The probability of being too low for this interval is    (from before)

\item The probability of being too high for this interval is    (from before)

\item Therefore the probability of being outside the interval is  0.0359 + 0.0250 = 0.0609.

\item Therefore the probability of being inside the interval is 1- 0.0609 = 0.9391

  = 0.9391
\end{itemize}

\end{framed}


\item Suppose 40\% of employees in a large company favour unionisation.  A poll of 10 employees in this company is taken.  
 \begin{itemize}
\item[(i)]  What is the probability that 4 or more employees polled favour unionisation? 
\item[(ii)]  What is the probability that less than 2 employees polled favour unionisation?
\item[(iii)]  What is the probability that exactly 5 employees polled favour unionisation?
\item[(iv)]  What is the mean and variance for this distribution?
\end{itemize}	

\item A student is practising for an upcoming high jump event.  The height that she will clear each time she jumps is normally distributed with a 
mean of 72 inches and a standard deviation of 4 inches.  
\begin{itemize}
\item[(i)]  What is the probability that the jumper will clear 76 inches or higher on a single jump?
\item[(ii)]  What is the probability that the height she jumps is between 68 and 76 inches on a single jump?
\item[(iii)]  What is the minimum height she must jump in order for the jump to be in the highest 10%?
\end{itemize}
    
\item Telephone calls coming in to a busy switchboard follow a Poisson distribution with 3 calls expected in a one minute period.  
The switchboard operator can answer at most 3 calls in a one minute period; the fourth and succeeding calls receive a busy signal.
\begin{itemize} 
\item[(i)]  	   Find the probability of receiving a busy signal.
\item[(ii)]  	   The switchboard operator leaves the switchboard unattended for 2 minutes.  
\item[(iii)]  What is the probability that exactly 1 call will be missed during that 2 minute period?
\end{itemize}       

\item In what circumstances can the Poisson distribution be used to approximate the Binomial distribution?
      


\item Four  per cent of PCB boards purchased over the Internet from the Far East have some defect.  
From a large consignment of boards, 50 are chosen at random.  
What is the probability that:
\begin{itemize}
\item[(i)]  	3 or more boards have some defect?
\item[(ii)]  	Exactly 2 boards are defective?
\item[(ii)]  	Less than 3 boards are defective?
\item[(iv)]  	If more than 5 boards from the 50 were defective what action would you take? (justify)
\end{itemize} 
\item Flaws occur in a hard wood timbers  at the rate of 1.5 per linear  metre section.  Calculate the probability that:
\begin{itemize}
\item[(i)]  	3 or more flaws will occur in a 3  metre length 
\item[(ii)]  	Exactly 4 flaws will occur in a 10 metre length
\item[(iii)]  	8 or less flaws will occur in a 6  metre length 
\end{itemize} 

\item There is a constant probability of 0.05  that the power supply in telecoms network will not start.  
You are requested to calculate the probability that the power supply will fail the 5th time it is activated.



\end{enumerate}

\end{document}
