\documentclass[]{article}
\voffset=-1.5cm
\oddsidemargin=0.0cm
\textwidth = 470pt
\usepackage[utf8]{inputenc}
\usepackage[english]{babel}
\usepackage{framed}

\usepackage{multicol}
\usepackage{amsmath}
\usepackage{amssymb}
\usepackage{enumerate}
\usepackage{multicol}
\begin{document}
\section*{Graph Theory: Tutorial Sheet}

\section*{Invertible Functions}

\begin{enumerate}
    \item What conditions must be satisfied for a function to have an inverse.
\begin{enumerate}[(a)]
\item One-one and onto
\item One-to-one only
\item onto only
\item Neither onto nor One-to-One 
\end{enumerate}


\item If f is a function for which the rule is f(x) = 7/8  - x, where x is real, the rule for the inverse function f-1 is:
\begin{enumerate}[(a)]
\item	$f^{-1}(x)$ = 8/7 + x
\item 	$f^{-1}(x)$ = -8/x + 7 
\item 	$f^{-1}(x)$ = 2x + 73/4 
\item 	$f^{-1}(x)$ = 7/8 - x % (This one)
\item 	$f^{-1}(x)$ = 8/7 + x
\end{enumerate}

\item Which of the following functions is not one-to-one?
\begin{enumerate}[(a)]
\item	$f(x) = 9 - x2, x \geq 0$
\item 	$f(x) = 1/x^2  - 9$ %- (This one)
\item 	$f(x) = 1 -9x$
\item 	$f(x) = \sqrt{x}$
\item	$f(x) = 3/x$
\end{enumerate}

\item 
The range of the function with rule $f(x) = \|x - 4\| + 3$ is:
\begin{enumerate}[(a)]
\item	$(4, \infty)$
\item	$\mathbb{R}$
\item 	$[3, \infty)$ %- This one
\item 	$(4, \infty)$
\item	$(-1, \infty)$
\end{enumerate}


\item A function $f : X \rightarrow Y$, where $X = \{p,q,r,s\}$ and $Y=\{1,2,3,4,5\}$ is given by the subset of $ X \times Y$ ,
i.e. $\{(q,3),(r,3),(p,5),(s,2)\}$.

\begin{itemize}
\item[i.] Show $f$ as an arrow diagram.
\item[ii.] State the domain, the co-domain and range of $f$.
\item[iii.] Say why $f$ does not have the \textbf{one-to-one} property and why $f$ does not have the \textbf{onto} property, giving a specific counter example in each case.
\end{itemize}
\begin{framed}
\textbf{Solutions}
\begin{itemize}
\item[i.] (Done on whiteboard)
\item[ii.] Domain, Co-Domain and Range
\begin{itemize}
\item[Domain] $\{a,b,c,d\}$,
\item[Co-Domain] $\{1,2,3,4,5\}$
\item[Range] $\{2,3,5\}$
\end{itemize}
\end{itemize}

\begin{itemize}
\item Onto - Range is equivalent to Co-domain.
\item no element of co-domain unused.
\item one to one - Each element of domain has one image in the co-domain. Each image has only one ancestor.
\end{itemize}
\end{framed}

\item Consider the functions $f: \mathcal{R} \rightarrow \mathcal{Z}$ and $g: \mathcal{r} \rightarrow \mathcal{R}$ given by
\[f(X) = \rfloor x-1 \lfloor \]
\[g(X) = | x-1 | \]

\begin{itemize}
\item[i.] Write down the domain, co-domain and range of f and g(x). 
\item[ii.] For each function, say whether or not it is one to one, justifying your
answer. 
\item[iii.] For each function, say whether or not it is onto, justifying your answer.
\end{itemize}

\item 
State whether or not each of the following functions has an inverse, justifying your answer. In the cases Where there is an inverse define it fully.

\begin{itemize}
\item[(i)] $f : S \rightarrow \mathcal{Z}^{+} $ defned in part (a) (see Section 2.9).
\item[(ii)] $g : \mathcal{R} \rightarrow  \mathcal{R} $ defined by $g(x) = x^2$.
\item[(iii)] $h : \mathcal{R} \rightarrow  \mathcal{R} $ defined by $h(x) = 4x - 1. $
\end{itemize}
\begin{framed}
\textbf{Solutions}
\begin{itemize}
\item No Inverse. Function is not onto, only one-to-one . Each name has only one image But each number can have more than one ancestor. Also the Co-domain and Range must be assumed to be not equal. Even very very long names do not exceed 200 letters.
\item No Inverse
\item Inverse Exists 
\[h^{-1}(x) = \frac{x+1}{4}  \]
\end{itemize}
\end{framed}


\item 
\begin{enumerate}
\item  Given a real number $x$, say how $\lfloor x \rfloor$, the floor of $x$, is defined.

\item  The function $f : \mathcal{R} \rightarrow \mathcal{R}$ is given by the rule
\[f(x) \lfloor x/2 \rfloor\]

\begin{itemize}
\item[ i.] Find $f(-3)$ and $f(3)$ 
\item[ii.] Justifying your answer, say whether f is one-to-one.
\item[iii.], Justifying your answer, say whether f is onto.
\end{itemize}
\end{enumerate}


\begin{framed}
\textbf{Solutions}
\begin{itemize}
\item f(3) = 0 and f(-3) = - 1
\item No, Different members of Domain can take the same value.
\item No, $f$ takes on integer values only while the codomain
is specified as $\mathcal{R}$.
\end{itemize}

\end{framed}

\item 


For each of the following equations, give \textbf{two} different examples of a real number $x$ which 
satisfies the equations:
\begin{itemize}
\item $\lfloor x \rfloor = 3 $
\item $\lceil x \rceil = -1 $
\item $| x-5 | =12 $
\end{itemize}

\begin{framed}
\textbf{Solutions}
\begin{itemize}
\item $\lfloor x \rfloor = 3 $ : two examples $\pi$ and $3.5$
\item $\lceil x \rceil = -1 $ : two examples -1.2 , -1.6
\item $| x-5 | =12 $  Answers: -7 and 17
\end{itemize}
\end{framed}

\item 
Given any number $x in \mathcal{R}$ the floor value is denoted $\rfloor x \lfloor$ and the absolute value is denoted by $|x|$.

\begin{itemize}
\item[a] Find $\rfloor \sqrt{2} \lfloor$ and $|-2|$.
\item[b] Find the set of values of a such that $\rfloor a \lfloor$ = 1, and the set of values $|b| = 1$
\item[c] (Discussed Separately)
\end{itemize}

\begin{framed}
\textbf{Part A}: 
\begin{itemize}
\item $\sqrt{2} = 1.414214 \ldots$
\item Floor function of x is the integer that precedes x.
\[\rfloor \sqrt{2} \lfloor = 1\]
\item The absolute value of -2 is simply 2.
\item The set of values of $a$ for which $\rfloor a \lfloor$ = 1 is all real numbers between 1 and 2. $a$ may take the value 1, but not the value $2$.
\[ 1 \leq a <2 , a \in \mathcal{R}\]
\item The set of values of $b$ for which $|b| = 1$ are simply the values -1 and 1.
\[ b= \{-1,1\}  b \in \mathcal{Z}\]
\end{itemize}
%--------------------------------------------------- %
\end{framed}
\item \begin{itemize}
\item [i.] State the condition to be satisified in order for a function to have an inverse.
\item [ii.] Given $ f : \mathcal{R} \rightarrow \mathcal{R} $ where $f(x) = 2x-1$, define fully the inverse function $f^{-1}$ 
and state the values of $f^{-1}(1)$
\item [ii.] Given $ g : \mathcal{R} \rightarrow \mathcal{R} $ where $g(x) = 3^x$, define fully the inverse function $g^{-1}$ 
and state the values of $g^{-1}(1)$
\end{itemize}
This requires giving the inverse function in algebraic terms and its domain and co-domain.


\item  Evaluate the following function for $x = 1,2$ and $5$ respectively.
	\[ f(x) = \frac{e^x + {e^{-x}}}{2} \]

%	Remark:  Example 1 was done in previous class.

\end{enumerate}

\end{document}
