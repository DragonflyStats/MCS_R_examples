\documentclass[]{article}
\voffset=-1.5cm
\oddsidemargin=0.0cm
\textwidth = 470pt
\usepackage[utf8]{inputenc}
\usepackage[english]{babel}
\usepackage{framed}

\usepackage{multicol}
\usepackage{amsmath}
\usepackage{amssymb}
\usepackage{enumerate}
\usepackage{multicol}

\begin{document}




\section*{ Hexadecimal Numbers - Tutorial Sheet}
%------------------------------------------------------------------------%
\begin{enumerate}
\item Calculate the decimal equivalent of the hexadecimal number $(A2F.D)_{16}$
\item Working in base 2, compute the following binary additions, showing all you workings
\[(1110)_2 + (11011)_2 + (1101)_2 \]


\item Working in base 2 perform the following calculation, showing all your working. 
\[110101_2 + 10111_2 - 100001_2\]
\item Express the following hexadecimal number as a decimal number: $(A32.C)_{16}$.

\item Convert the following decimal number into base 2, showing all your working:
$(253)_{10}$. 
\item Express the recurring decimal $0.4242424\ldots$
 as a rational number in its simplest
form.
\item  Express the following hexadecimal number as a decimal number: $(A32.8)_{16}$.
\item Convert the following decimal number into base 2, showing all your working:
$(253)_{10}$. [2]
\item  Express the recurring decimal $0.4242424\ldots$
as a rational number in its simplest
form. 
%---------------------------%
\item Suppose 2341 is a base-5 number
Compute the equivalent in each of the following forms:
\begin{itemize}
\item[(i)] decimal number
\item[(ii)] hexadecimal number
\item[(iii)] binary number
\end{itemize}
%---------------------------%


\item Answer the following questions about the hexadecimal number systems
\begin{itemize}
\item[a)] How many characters are used in the hexadecimal system?
\item[b)] What is highest hexadecimal number that can be written with two characters? \item[c)] What is the equivalent number in decimal form?
\item[d)] What is the next highest hexadecimal number?
\end{itemize}


\item Which of the following are not valid hexadecimal numbers?
  \begin{multicols}{4}
    \begin{itemize}
    \item[a)] $73$
    \item[b)] $A5G$
    \item[c)] $11011$
    \item[d)] $EEF	$
    \end{itemize}
  \end{multicols}

\item Express the following decimal numbers as a hexadecimal number.
  \begin{multicols}{4}
    \begin{itemize}
    \item[a)] $(73)_{10}$
    \item[b)] $(15)_{10}$
    \item[c)] $(22)_{10}$
    \item[d)] $(121)_{10}$
    \end{itemize}
  \end{multicols}


\item Compute the following hexadecimal calculations.
  \begin{multicols}{4}
    \begin{itemize}
    \item[a)] $5D2+A30$
    \item[b)] $702+ABA$
    \item[c)] $101+111$
    \item[d)] $210+2A1$
    \end{itemize}
  \end{multicols}
\item 
\begin{itemize}
	\item[(i)] Calculate the decimal equivalent of the hexadecimal number $(A2F.D)_{16}$
	\item[(ii)] Working in base 2, compute the following binary additions, showing all you workings
	\[(1110)_2 + (11011)_2 + (1101)_2 \]
	\item[(iv)] Express the recurring decimal $0.727272\ldots$ as a rational number in its simplest form.
\end{itemize}


	%---------------------------%
	\item Suppose 2341 is a base-5 number
	Compute the equivalent in each of the following forms:
	\begin{itemize}
		\item[(i)] decimal number
		\item[(ii)] hexadecimal number
		\item[(iii)] binary number
	\end{itemize}

\end{enumerate}
%========================================================================%

\end{document}
