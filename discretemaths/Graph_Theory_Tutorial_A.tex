\documentclass[]{article}
\voffset=-1.5cm
\oddsidemargin=0.0cm
\textwidth = 470pt
\usepackage[utf8]{inputenc}
\usepackage[english]{babel}
\usepackage{framed}

\usepackage{multicol}
\usepackage{amsmath}
\usepackage{amssymb}
\usepackage{enumerate}
\usepackage{multicol}
\begin{document}
\section*{Graph Theory: Tutorial Sheet}
\begin{enumerate}
\item Draw a graph with degree sequence 4,3,2,2. If it is not possible to draw this graph, explain why.
\item Draw a graph with degree sequence 4,3,3,2,2. If it is not possible to draw this graph, explain why.
\item Explain what is meant by a complete graph. How is a complete graph, with n vertices denoted?
\item How many edges does a complete graph with 8 vertices contain?
\item A graph is called $k$-regular if each of its vertices has degree $k$. Construct an 
example of: 
\begin{enumerate}[(i)]

\item a 2-regular graph with 5 vertices; 
\item a 3-regular graph with 6 vertices
\item a 4-regular graph with 8 vertices.
\end{enumerate}

\item Is it possible to construct an 8 vertex graph where each vertex is connected to exactly 5 vertices? Is it possible to do so for a 9 vertex graph?
\item Consider a d-regular graph on 7 vertices. What are the possible values for d. For each viable value for d, how many edges would there be?

\item 
Given the following definitions for simple, connected graphs:
\begin{enumerate}
\item $K_n$ is a graph on $n$ vertices where each pair of vertices is connected by an edge;
\item $C_n$ is the graph with vertices $v_1, v_2, v_3, \dots, v_n$ and edges $\{v_1,v_2\}, \{v_2,v_3\}, \dots\{v_n, v_1\}$;
\item $W_n$ is the graph obtained from $C_n$ by adding an extra vertex,$v_{n+1}$, and edges
from this to each of the original vertices in $C_n$.
\end{enumerate}
Draw $K_4$, $C_4$, and $W_4$. 

%-------------------------------------------------------- %

% Section 5 Graph Theory
\item 
Let G be a simple graph with vertex set $V(G) = \{v1, v2, v3, v4, v5\}$ and adjacency lists as follows:
\begin{verbatim}
v1 : v2 v3 v4
v2 : v1 v3 v4 v5
v3 : v1 v2 v4
v4 : v1 v2 v3.
v5 : v2
\end{verbatim}
\begin{enumerate}[(i)]
\item List the degree sequence of G. Draw the graph of G.
\item Find two distinct paths of length 3, starting at v3 and ending at v4. Find a 4 cycle in G.
\item Let G be a graph and let v be a vertex of G. Say what is meant by the degree of v.
\item State, without proving, a result connecting the degrees of the vertices of a graph G with the number of its edges.

\end{enumerate}
\item
\begin{enumerate}[(i)]

\item State, without proving, a result connecting the degrees of the vertices of 
a graph $G$ with the number of its edges. 
\item Use this result to find the number of edges of a 3-regular graph with 10 
vertices. 

\item Explain why it is not possible to construct a 3-regular graph with 9 vertices. 

\item A simple, connected graph has 7 vertices, all having the same degree $d$. State the possible values of $d$ and for each value also give the number of edges in the corresponding graph.
\item 
Another simple, connected graph has 6 vertices, all having the same degree; $n$. Draw such a graph when $n = 3$ and state the other possible values of $n$. 
\end{enumerate}
\end{enumerate}
\end{document}
