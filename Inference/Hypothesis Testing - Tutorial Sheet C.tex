
\documentclass[]{article}

\voffset=-1.5cm
\oddsidemargin=0.0cm
\textwidth = 480pt

\usepackage{framed}
\usepackage{subfiles}
\usepackage{graphics}
\usepackage{newlfont}
\usepackage{eurosym}
\usepackage{amsmath,amsthm,amsfonts}
\usepackage{amsmath}
\usepackage{enumerate}
\usepackage{color}
\usepackage{multicol}
\usepackage{amssymb}
\usepackage{multicol}
\usepackage[dvipsnames]{xcolor}
\usepackage{graphicx}
\begin{document}

\section*{Hypothesis Testing : Tutorial Sheet C}

\begin{enumerate}
\item 
63 of 100 first time convicts serving at least 3 years re-offended, whereas 70 of 140 first time convicts serving less than 3 years re-offended. 
\begin{enumerate}[(a)]
\item	Test the null hypothesis that the re-offending rate does not depend on the length of the first sentence at a significance level of 5%.
\item 	Calculate a 95\% confidence interval for the difference in the re-offending rates of these two groups.
\item 	Based on this confidence interval, is there any evidence that the re-offending rates differ? 
\end{enumerate}

\item The typing speeds for one group of eight IT students were recorded both at the beginning of
year 1 of their studies and at the end of year 4. The results (in words per minute) are given
below:
\begin{center}
\begin{tabular}{|c|c|c|c|c|c|c|c|c|} \hline 
  Subject & A& B& C& D& E& F& G& H \\ \hline
Year 1& 173& 183& 176& 191& 184& 177& 175& 177 \\ \hline
Year 4& 172& 184& 180& 190& 191& 187& 181& 183 \\ \hline
\end{tabular}
\end{center}
\item 
%(c) Inference Procedures (9 Marks)
A study finds that 45\% of IT users out of a random sample of 500 in a
large community preferred one web browser to all others. In another large
community, 30\% of IT users out of a random sample of 390 prefer the
same web browser.
\begin{itemize}
    \item[(a)] Compute the point estimate for the difference in proportions
of IT users who prefer this particular web browser.
\item[(b)] Compute a 95\% confidence interval for this difference in
proportions.
\item[(c)] Based on this confidence interval, test the hypothesis that
the proportion of IT users using this web browser is the same for both
communities. State your null and alternative hypotheses clearly.
\end{itemize}


\item 
% Hypothesis Testing (9 Marks)
In a computer hardware manufacturing plant, machine X and machine Y
produce identical components. The management investigate whether or
not there is a difference in the mean diameter of the components from
both machines.
\begin{itemize}
\item A random sample of 144 components from machine X had a mean of
36.38 mm and a standard deviation of 3.0 mm.
\item  A random sample of 225 components from machine Y had a mean of
36.88 mm and a standard deviation of 2.8 mm.
\end{itemize}
Compute a 95\% confidence interval for the difference in means.What is your conclusion for this procedure? Justify your
answer.
\item Two similar machines are making components of a particular length in mm.
Compute a 95\% confidence interval for the difference in average length between the components

The results for each machine are
\begin{itemize}
\item Machine A: $\{23.7, 23.0, 22.2, 24.0, 21.2, 23.1, 27.1, 24.0\}$ $(n_1 = 8)$
\item Machine B: $\{23.4, 15.3, 30.9, 18.8, 25.3, 25.2, 32.1\}$ ($n_2 = 7)$
\end{itemize}

You may assume that the populations are normal with variance 9 for the first machine and 16 for
the second machine.

\item 
In a computer hardware manufacturing plant, machine X and machine Y
produce identical components. The management investigate whether or
not there is a difference in the mean diameter of the components from
both machines.
\begin{itemize}
\item A random sample of 144 components from machine X had a mean of
36.38 mm and a standard deviation of 3.0 mm.
\item  A random sample of 225 components from machine Y had a mean of
36.88 mm and a standard deviation of 2.8 mm.
\end{itemize}
A hypothesis test was used to determine whether or not the means are
significantly different. A 5\% significance level was used.
\begin{enumerate}[(a)]
\item What is the null and alternative hypothesis?
\item Compute the test statistic.
\item What is your conclusion for this procedure? Justify your
answer.
\end{enumerate}

	\item Two similar machines are making components of a particular length
	in mm. Give a $95\%$ confidence interval for the difference in average length between
	the components produced by the two machines. Assume that the populations are normal
	with variance 9 for the first machine and 16 for the second machine.
	\begin{itemize}
	\item The results for each machine are\\ \bigskip
	\textbf{Machine A:} $\{ 23.7, 23.0, 22.2, 24.0, 21.2, 23.1, 27.1, 24.0 \}   $ ($n_1= 8$)\\ \bigskip
	\textbf{Machine B:} $\{23.4, 15.3, 30.9, 18.8, 25.3, 25.2, 32.1\}$  ($n_2=7$) \\
\end{itemize}


%==============================================================%
\item 
63 of 100 first time convicts serving at least 3 years re-offended, whereas 70 of 140 first time convicts serving less than 3 years re-offended. 
\begin{enumerate}[(a)]
\item	Test the null hypothesis that the re-offending rate does not depend on the length of the first sentence at a significance level of 5%.
\item 	Calculate a 95\% confidence interval for the difference in the re-offending rates of these two groups.
\item 	Based on this confidence interval, is there any evidence that the re-offending rates differ? 
\end{enumerate}

%==============================================================%
\item 
A study was made of children who were hospitalised as a result of a car accident. 
\begin{itemize}
	\item 290 of the children were not wearing seat belts and 50 of these were seriously injured. 
	\item 123 children wore seat belts and 16 were seriously injured. 
	
\end{itemize}

Test the hypothesis that the rate of serious injury is the same for children who wear a seat belt or not. Clearly state your null and alternative hypotheses and your conclusion.

\end{enumerate}
\end{document}

