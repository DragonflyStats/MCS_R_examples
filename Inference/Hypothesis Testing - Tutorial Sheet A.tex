
\documentclass[]{article}

\voffset=-1.5cm
\oddsidemargin=0.0cm
\textwidth = 480pt

\usepackage{framed}
\usepackage{subfiles}
\usepackage{graphics}
\usepackage{newlfont}
\usepackage{eurosym}
\usepackage{amsmath,amsthm,amsfonts}
\usepackage{amsmath}
\usepackage{enumerate}
\usepackage{color}
\usepackage{multicol}
\usepackage{amssymb}
\usepackage{multicol}
\usepackage[dvipsnames]{xcolor}
\usepackage{graphicx}
\begin{document}

\section*{Hypothesis Testing - Tutorial Sheet A}

\begin{enumerate}
    \item \textbf{Hypothesis Testing Procedure}
\begin{enumerate}[(a)]    
\item State the four steps of hypothesis testing.

\item State the general structure for a test statistic.
\end{enumerate}
%============================%

\item \textbf{Dixon Q Test For Outliers}
\\ 
The typing speeds for one group of 12 Engineering students were recorded both at the beginning 
of year 1 of their studies. 
The results (in words per minute) are given below: 

\[\{121, 146, 150, 149, 142, 170, 153, 137, 161, 156, 165,  137, 178, 159\} \]

Use the Dixon Q-test to determine if the lowest value (121) is an outlier. 
You may assume a significance level of 5\%. 
\begin{enumerate}[(a)]
\item Formally state the null hypothesis and the alternative hypothesis. \item Compute the test statistic. 
\item By comparing the test statistic to the appropriate critical value, state your conclusion for this test. 

\end{enumerate}

%=================%


\item The average height of a sample of 16 students was 173cm with a variance of 144cm$^2$. 
The average height of the Irish population is 169cm. 
\begin{enumerate}[(a)]
\item Can it be stated at a significance level of 5\% that students are on average taller than the population as a whole? 
 \item What assumption is used to carry out this test? Is this assumption reasonable?
\end{enumerate}

\item The government wishes to increase the proportion of people taking government training courses who obtain a job in the following 3 months. Before they introduced the new schemes this figure was 58\%. A survey of 300 people who took the new courses indicated that 188 of them gained a job. A government official stated that this indicates that the new courses have been more successful. 
Is his statement reasonable at a significance level of 5\%?

%=================%
\item Mean blood iron concentration for children with adequate nutrition is taken to be 110mg/dl. \\ 25 randomly selected children from a disadvantaged urban area were given blood tests. The mean concentration of iron from this sample was 98 mg/dl with a standard deviation of 25.5 mg/dl.
\begin{itemize}
	\item[(a)] (4 Marks) Calculate a 95\% confidence interval for the mean concentration of iron for children in this area. 
	\item[(b)](2 Marks) Interpret this confidence interval.  Do these results provide evidence that children in this area suffer from iron deficiency? 
\end{itemize}
\medskip
Test this hypothesis using a 5\% level of significance. 

\begin{itemize}
	\item[(c)](3 Marks) Formally state your null and alternative hypotheses.
	\item[(d)](4 Marks) Compute the test statistic.
	\item[(e)](2 Marks) Discuss your conclusion to this test, supporting your statement with reference to appropriate values.
\end{itemize}

\item For a random sample of 10 light bulbs for a particular brand, the mean bulb life is 4,000 hr with a sample
standard deviation of 200 hours. For another brand of bulbs, a random sample of 8 has a sample mean lifetime of 4,300 hours and a sample
standard deviation of 250 hours.\\

\begin{itemize}
	\item[(a)]Test the hypothesis that there is no difference between the mean operating life of the two brands of bulbs,
using the 5 percent level of significance.
\end{itemize}


\item
An exercise physiologist wants to determine if several short bouts of exercise provide the same benefit for cardiovascular fitness as one long bout of exercise. \\ \smallskip

\noindent 50 volunteers are randomly assigned to group 1 and do standardized aerobic exercise on a stationary bicycle for 30 minutes once a day, 5 days a week. 40 volunteers are randomly assigned to group 2 and do the same exercise for 10 minutes, 3 times a day, 5 days a week. Cardiovascular fitness was measured by VO2 max (maximum oxygen consumption while exercising). 

\begin{description}
	\item[Group 1] The mean change in VO2 after 12 weeks of exercise was 2.1 for group 1 with a standard deviation of 1.7.
	\item[Group 2] The mean change in VO2 after 12 weeks of exercise was 0.7 for group 2 with a standard deviation of 1. 
\end{description}

\noindent Test the hypothesis that there is no significant difference between two groups are the same.
	
\begin{itemize}
	\item[(a)](3 Marks) Formally state your null and alternative hypotheses.
	\item[(b)](4 Marks) Compute the test statistic.
	\item[(c)](3 Marks) Discuss your conclusion to this test, supporting your statement with reference to appropriate values.
\end{itemize}
%=================%


\end{enumerate}


\end{document}

