\documentclass[]{report}

\voffset=-1.5cm
\oddsidemargin=0.0cm
\textwidth = 480pt

\usepackage{framed}
\usepackage{subfiles}
\usepackage{graphics}
\usepackage{newlfont}
\usepackage{eurosym}
\usepackage{enumerate}
\usepackage{amsmath,amsthm,amsfonts}
\usepackage{amsmath}
\usepackage{color}
\usepackage{amssymb}
\usepackage{multicol}
\usepackage[dvipsnames]{xcolor}
\usepackage{graphicx}
\begin{document}
%====================================================================================================%

\subsection*{Exercise: Combinations}

A committee of 4 must be chosen from 3 women and 4 men.

\begin{enumerate}[(1)]
\item In how many ways can the committee be chosen, regardless of sex.
\item In how many cans 2 men and 2 women be chosen.
\item Compute the probability of a committee of 2 men and 2 women are chosen.
\item Compute the probability of at least two women.
\end{enumerate}

%===================================================================================================%
\subsection*{Part 1}

We need to choose 4 people from 7:

This can be done in

\[
^7 C_4  = {7!  \over 4! \times 3!} =  {7 \times 6 \times 5 \times 4!  \over 4! \times 3!} = 35 \mbox{ ways.}
\]

%===================================================================================================%
\subsection*{Part 2}

With 4 men to choose from, 2 men can be selected in \[
^4 C_2  = {4!  \over 2! \times 2!} =  {4 \times 3 \times 2!  \over 2! \times 2!} = 6\mbox{ ways.}
\]

Similarly 2 women can be selected from 3 in
\[
^3 C_2  = {3!  \over 2! \times1!} =  {3 \times 2!  \over 2! \times 1!} = 3\mbox{ ways.}
\]

%===================================================================================================%

Thus a committee of 2 men and 2 women can be selected in $ 6 \times 3  = 18 $ ways.\\

\subsection*{Part 3}

The probability of two men and two women on a committee is
\[ {\mbox{Number of ways of selecting 2 men and 2 women} \over \mbox{Number of ways of selecting 4 from 7}} = {18 \over 35 }\]


%===================================================================================================%
\subsection*{Part 4}
\begin{itemize}
\item The probability of at least two women is the probability of 2 women or 3 women being selected.
\item We can use the addition rule, noting that these are two mutually exclusive events.
\item From before we know that probability of 2 women being selected is 18/35.
\item We have to compute the number of ways of selecting 1 man from 4 (4 ways) and the number of ways of selecting three women from 3 (only 1 way)
\item The probability of selecting three women is therefore ${4 \times 1 \over 35} = 4/35$
\item So using the addition rule
\[ Pr(\mbox{ at least 2 women }) = Pr(\mbox{ 2 women }) + Pr(\mbox{ 3 women }) \]
\[ Pr(\mbox{ at least 2 women })  = 18/35 + 4/35 = 22/35 \]
\end{itemize}

%===================================================================================================%


\end{document}
