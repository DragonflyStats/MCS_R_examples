

\documentclass[]{article}
\voffset=-1.5cm
\oddsidemargin=0.0cm
\textwidth = 470pt
\usepackage[utf8]{inputenc}
\usepackage[english]{babel}
\usepackage{framed}

\usepackage{multicol}
\usepackage{amsmath}
\usepackage{amssymb}
\usepackage{enumerate}
\usepackage{multicol}




%opening
\title{Set Theory - Tutorial Sheet}


\begin{document}
\begin{enumerate}
\item 
For each of the following sets, write out the set using the listing method.
Also write down the cardinality of each set.

\begin{enumerate}[(a)]
\item $\{ s : s $ is an negative integer $ -10 \leq s \leq 0 \}$
\item $\{ t : t $ is an even number $ 1 \leq t \leq 20 \}$
\item $\{ u : u $ is a prime number $ 1 \leq u \leq 20 \}$
\item $\{ v : v $ is a multiple of 3 $ 1 \leq v \leq 20 \}$
\end{enumerate}
\item 
Consider the set $Z$:
\[ Z = \{ a,b,c\}  \]
\begin{itemize}
\item[(i)] How many sets are in the power set of $Z$? 
\item[(ii)] Write out the power set of $Z$. 
\item[(iii)] How many elements are in each element set?
\end{itemize}
%%%%%%%%%%%%%%%%%%%%%%%%%%%%%%%%%%
\item Describe the following set by the listing method
\[ \{ 2r+1 : r \in \mathbb{Z^{+}} \mbox{ and } r \leq 5  \} \]
\item Describe the following set by the listing method
\begin{enumerate}[(a)]
\item $\{ s :  \mbox{s is an odd integer and} 2 \leq s \leq 10 \}$
\item $\{ 2m :  m \in Z \mbox{ and } 5 \leq m \leq 10 \}$
\item $\{ 2^t :  t \in Z \mbox{ and } 0 \leq t \leq 5 \}$
\end{enumerate}
%%%%%%%%%%%%%%%%%%%%%%%%%%%%%%%%%%
\item Describe the following set by the builder method
\begin{enumerate}[(a)]
\item \{12,13,14,15,16,17\}
\item \{0,5,-5,10,-10,15,-15,.....\}
\item \{6,8,10,12,14,16,18\}
\end{enumerate}
%%%%%%%%%%%%%%%%%%%%%%%%%%%%%%%%%%


\item Consider the universal set $U$ such that
\[U=\{1,2,3,4,5,6,7,8,9\} \] 
and the sets
\[A=\{2,5,7,9\} \] 
\[B=\{2,4,6,8,9\} \]

\begin{multicols}{2}
\begin{itemize}
\item[(a)] $A-B$
\item[(b)] $A \otimes B$
\item[(c)] $A \cap B$
\item[(d)] $A \cup B$
\item[(e)] $A^{\prime} \cap B^{\prime}$
\item[(f)] $A^{\prime} \cup B^{\prime}$
\end{itemize}
\end{multicols}


% 2007 Q8
\item Given S is the set of all 5 digit binary strings, E is the set of a 5 digit
binary strings beginning with a 1 and F is the set of all 5 digit binary strings ending
with two zeroes.
\begin{itemize}
\item[(a)] Find the cardinality of S, E and F.
\item[(b)] Draw a Venn diagram to show the relationship between the sets S, E and F.
Show the relevant number of elements in each region of your diagram.
\end{itemize}


\item Suppose we have the sets A and B defined as follows:
\[ A = \{ \sqrt{2}, \frac{3}{2}, 2 \}\]
\[  B = \{ x \in R :  X not in Q \}  \]


\begin{enumerate}[(i)]
\item $A \cap Q$
\item $A \cap B$
\item $B \cup Q$
\end{enumerate}


\item Shade in the following areas on Venn diagrams.

\begin{multicols}{2}
\begin{itemize}

\item[(a)] $A^\prime\; \cup\; B$

\item[(b)] $A \cap\; B^\prime\;$

\item[(c)] $(A \cap\; B)^\prime\;$

\item[(d)] $A^\prime\; \cup\; B^\prime\;$

\item[(e)] $(A \cup\; B)^\prime\;$

\item[(f)] $A^\prime\; \cap\; B^\prime\;$
\end{itemize}
\end{multicols}

%%%%%%%%%%%%%%%%%%%%%%%%%%%%%%%%%%%%%%%%%%%%%%%%%%%%%%%%%%%%%

\item Draw a Venn Diagram to represent the universal set
$\mathcal{U} = \{0,1,2,3,4,5,6\}$ with subsets:\\
\[A = \{2,4,5\} \]
\[B = \{1,4,5,6\} \]

\noindent Find each of the following
\begin{multicols}{2}
\begin{itemize}
\item[(a)] $A \cup B $
\item[(b)] $A \cap B $
\item[(c)] $A-B$
\item[(d)] $B-A$
\item[(e)] $A \otimes B$
\end{itemize}
\end{multicols}



\item
Given $S$ is the set of all 5 digit binary strings, $E$ is the set of a 5 digit
binary strings beginning with a 1 and F is the set of all 5 digit binary strings ending
with two zeroes.
\begin{itemize}
\item[(a)] Find the cardinality of S, E and F.
\item[(b)] Draw a Venn diagram to show the relationship between the sets S, E and F.
Show the relevant number of elements in each region of your diagram.
\end{itemize}

\item 
Using membership tables
\begin{center}
    \begin{tabular}{|ccc|c|c|c|}
	\hline
	% after \\: \hline or \cline{col1-col2} \cline{col3-col4} ...
	A & B & C & x & y & z \\\hline
	0 & 0 & 0 & 1 & 1 & 1 \\
	0 & 0 & 1 & 0 & 0 & 1 \\
	0 & 1 & 0 & 0 & 0 & 1 \\
	0 & 1 & 1 & 0 & 0 & 1 \\
	1 & 0 & 0 & 1 & 0 & 1 \\
	1 & 0 & 1 & 1 & 0 & 1 \\
	1 & 1 & 0 & 0 & 0 & 1 \\
	1 & 1 & 1 & 1 & 0 & 1 \\
	\hline
\end{tabular}
\end{center}
\begin{itemize}
	\item[(i)] Draw a venn diagram to show three subsets A,B and C of a universal set U intersecting in
	the most general way?
	\item[(ii)] How are sets $X$ and $Z$ related?
	\item[(iii)] Can you describe each of the subsets X,Y and Z in terms  of the
	sets A,B,C using the operations union intersection and set complement.
\end{itemize}


\item Describe the following set by the listing method
\[ \{ 2r+1 : r \in Z^{+} and r \leq 5  \} \]


\item 
Let n be an element of the set $\{10, 11, 12, 13, 14, 15, 16, 17, 18, 19\}$,
and p and q be the propositions:
p : n is even, $q : n > 15$.
Draw up truth tables for the following statements and find the values of n for
which they are true:
\begin{enumerate}
\item $p \vee \neg q$
\item $\neg p \wedge q$
\end{enumerate}

\end{enumerate}
\end{document}
