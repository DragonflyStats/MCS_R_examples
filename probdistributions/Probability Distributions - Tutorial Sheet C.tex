

\documentclass[]{report}

\voffset=-1.5cm
\oddsidemargin=0.0cm
\textwidth = 480pt

\usepackage{framed}
\usepackage{subfiles}
\usepackage{graphics}
\usepackage{newlfont}
\usepackage{eurosym}
\usepackage{amsmath,amsthm,amsfonts}
\usepackage{amsmath}
\usepackage{enumerate}
\usepackage{color}
\usepackage{amssymb}
\usepackage{multicol}
\usepackage[dvipsnames]{xcolor}
\usepackage{graphicx}
\begin{document}

\subsection*{Probability Distributions : Tutorial Sheet C}
\begin{enumerate}

\item 
Assume that the amount of wine poured into a bottle has a normal distribution with a mean of 750ml and a variance of 144ml$^2$.

\begin{itemize}

\item[(i)]  Calculate the probability that a bottle contains more than 765ml. (2 marks)
\item[(ii)]    Calculate the probability that a bottle contains between 744ml and 759ml. (3 marks)
\end{itemize}
%=========================================================== %
\item A machine fills bags with animal feed. The nominal weight of a bag is 50kg.
Because random variations the weight of a filled bag is normally distributed
$N(\mu, \sigma^2)$. The variance ($\sigma^2$) is known to be 0.01kg$^2$ and $\mu$ is set by the
operator to a particular value.

\begin{itemize}
\item[(i)] If ?  = 50kg calculate the probability of a bag containing less than
49.95kg?
\item[(ii)] Calculate the value of "?" such that only $2\%$ of the output are under the
nominal weight?
\end{itemize}
%--------------------------------------------------%

\item During the day, cars pass along a point on a remote road at an average rate of one per 20 minutes. Calculate the probability that 
\begin{itemize}
\item[(i)]in the course of an hour no car passes. 
\item[(ii)]in the course of 30 minutes exactly 4 cars pass
\item[(iii)]in the course of 30 minutes at least two cars pass
\end{itemize}




\item A computer server breaks down on average once every three months.

\begin{itemize}
\item What is the probability that the server breaks down three times in a quarter?
\item What is the probability that a server breaks down exactly five times in one year?
\end{itemize}

\item 
Statistical records for road traffic accidents on a particular stretch of road state that the average number of accidents per week is 2.

\begin{itemize}
\item Four accidents during a randomly selected week
\item No accidents
\end{itemize}

\item  Suppose calls come into a call centre randomly at a rate of one per 30 seconds.
\begin{enumerate}[(i)]
\item What is the distribution of the time to the second call?
\item Using this distribution, calculate the probability that the second call arrives within a minute. 
\item Using the appropriate discrete distribution, calculate the probability that at least 2 calls are received in a minute (note this probability has to be the same as above).
\item What is the exact distribution of the time to the 200th call?
\item Using the central limit theorem, give the normal distribution which approximates the distribution from iv).
\item Using your answer from v), estimate the probability that the time to the 200th call is less than 102 minutes.
\end{enumerate}

\item The average lifespan ppf a laptop is 5 year. You may assume that the lifespan of laptop computers follows an exponential distribution.

\begin{enumerate}
\item What is the probability that the lifespan of the laptop will be at least 6 years.
\[e^{-6/5} = 0.3011942\]
\item What is the probability that the lifespan of the laptop will not exceed 4 years.
\[e^{-4/5} = 0.449329\]
\item What is the probability that the lifespan of the laptop will be between 5 years and 6 years.
\[e^{-5/5} = 0.3678794\]
\end{enumerate}

\item 
A particular brand of hard disk is designed to last an average of 2 years. 
Assume that its lifetime is $T \sim \mbox{Exponential}(\lambda)$.\\
\begin{itemize}
\item[(a)] What is the value of $\lambda$?  \item[(b)] What is $Sd(T)$?  \item[(c)] Calculate $\Pr(T > 1)$.   \item[(d)] Calculate $\Pr(T < 5)$.  \item[(e)] Calculate $\Pr(2 < T < 5)$.  \item[(f)] Calculate the value of $t$ such that 80\% of hard disks fail before this time, i.e., $\Pr(T > t) = 0.2$.

\end{itemize}


\item 
Let $X \sim \text{Exponential}(\lambda=0.02)$. Calculate the following:\\
\begin{itemize}
\item[(a)] $\Pr(\,\overline{\!X} > 55)$ in a group of 100.  \item[(b)] $\Pr(\,\overline{\!X} < 53)$ in a group of 40.  \item[(c)] The value of $\bar x$ such that $\Pr(\,\overline{\!X} > \bar x) = 0.1$ when $n=65$.  \item[(c)] The value of $n$ if $\Pr(\,\overline{\!X} < 49) = 0.1$.
\end{itemize}

\item 
The \emph{average time} between customers arriving to a shop is 5 minutes. We will assume that the time, $T$, has an exponential distribution. Calculate the following:\\
\begin{itemize}
\item[(a)] The average arrival \emph{rate}, i.e., $\lambda$ customers per minute.  \item[(b)] The probability that we wait more than 15 minutes for the next customer.  \item[(c)] The probability that the next customer arrives within 1 minute.  \item[(d)] The average \emph{number of customers} in a 1 hour period. What is the standard deviation that goes with this average?  \item[(e)] The probability that \emph{15 or more} customers arrive in a 1 hour period.
\end{itemize}
\end{enumerate}
\end{document}

